\section{Specyfikacja}

\begin{flushleft}
	\hspace{1cm}Skrypt powstał z wykorzystaniem języka programowania \textbf{python} w wersji: \textbf{3.11.3} przy pomocy bibliotek: numpy, math, argparse, obsługiwany w systemie oprogramowania \textbf{Windows}. Program przyjmuję i zwraca dane w formie pliku z rozszerzeniem \textbf{.txt}
\end{flushleft}

\section{Przebieg ćwiczenia}

\begin{flushleft}
	\hspace{1cm}Do stworzenia programu wykorzystano funkcje wykonujące transformacjie stworzone na potrzeby Geodezji Wyższej I w semestrze 3. Wszystkie funckje zostały użyte jako metody pod klasą: Transformacje w celu umożliwenia sprawnego przeliczania współrzędnych z wykorzystaniem różnych modeli elipsoid. W tym celu wykorzystano metodę \textbf{init}, a zmienne zależne od elipsoid zapisano z wykorzystaniem odwołania \textbf{self}, tak aby metody mogły "pobrać" parametry elipsoidy wybranej przez użytkownika. W celu zwiększenia czytleności kodu zastosowano wyrażenie warunkowe: \textbf{if name == "main":}. W przypadku wprowadzenia nieprawidłowego modely elipsoidy program wyśle informację zwrotną z popełnionym błędem. Wykorzystano również bibliotekę argparse, dzięki której program może być wykorzystany z poziomu wiersza poleceń. Usprawnienie pracy i koordynacji w zespole było możliwe poprzez wykorzystanie wirtualnego repozytorium z portalu git.hub, które umożliwiało połączenie lokalnych repozytorium.
\end{flushleft}

\section{Napotkane przeciwności losu}

\begin{flushleft}
	\hspace{1cm} W celu poprawnego wczytania danych do programu plik .txt musi mieć odpowiedni format
\end{flushleft}