\section{Specyfikacja}

\begin{flushleft}
	\hspace{1cm}Skrypt powstał z wykorzystaniem języka programowania \textbf{python} w wersji: \textbf{3.11.3} przy pomocy bibliotek: numpy, math, argparse, obsługiwany w systemie oprogramowania \textbf{Windows}. Program przyjmuję i zwraca dane w formie pliku z rozszerzeniem \textbf{.txt}
\end{flushleft}

\section{Przebieg ćwiczenia}

\begin{flushleft}
	\hspace{1cm}Do stworzenia programu wykorzystano funkcje wykonujące transformacjie stworzone na potrzeby Geodezji Wyższej I w semestrze 3. Wszystkie funckje zostały użyte jako metody pod klasą \textbf{Transformacje} w celu umożliwenia sprawnego przeliczania współrzędnych z wykorzystaniem różnych modeli elipsoid. W tym celu wykorzystano metodę \textbf{init}, a zmienne zależne od elipsoid zapisano z wykorzystaniem odwołania \textbf{self}, tak aby metody mogły "pobrać" parametry elipsoidy wybranej przez użytkownika. W celu zwiększenia czytleności kodu zastosowano wyrażenie warunkowe: \textbf{if name == "main":}. Aby umożliwić wywołanie programu z poziomu wiersza poleceń wykorzystaliśmy bibliotekę \textbf{argparse}, pozwala ona na definiowanie oraz dodawanie arguentów do wywołania w konsoli. W przypadku wprowadzenia nieprawidłowego modely elipsoidy program wyśle informację zwrotną z popełnionym błędem.\\
	\hspace{1cm}Podczas tworzenia programu napotkaliśmy problem z odczytem współrzędnych wielu punktów z pliku. W przypadku gdy użytkownik na końcu pliku ze współrzędnymi wejściowymi utworzył zbyt dużo pustych linii otrzymywaliśmy błąd zamiany zmiennej typu string na float. Problem ten udało nam się rozwiązać stosując klauzulę \textbf{try...except}. Dzięki zastosowaniu tej klauzuli udało się również uniknąć błędnego działania programu w przypadku gdy użytkownik utworzy puste linie w dowolnym miejscu w pliku ze współrzędnymi wejściowymi.\\
	\hspace{1cm}Po rozwiązaniu problemu z pustymi liniami wpływającymi na błędne działanie programu udało nam się znaleźć problem, przez który nasz program omijał punkty w przypadku, gdy użytkownik wprowadzi dane w niewłaściwy sposób. Jeżeli w dokumencie tekstowym ze współrzędnymi wejściowymi pojawiła się dowolna ilość spacji przed współrzędnymi lub współrzędne były oddzielone więcej niż jedną spacją nasz program w celu uniknięcia błędu działania omijał ten punkt. W celu rozwiązania tego problemu postanowiliśmy zmienić format podawania współrzędnych w pliku tekstowym na taki, żeby współrzędne oddzielane były od siebie średnikami. Dzięki temu, niezależnie od tego czy w pliku wejściowym będą znajdować się jakiekolwiek spacje nasz program będzie w stanie je przeliczyć.
\end{flushleft}

