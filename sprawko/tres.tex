\section{Podsumowanie}

\begin{flushleft}
	\hspace{1cm}Podczas pracy nad projektem mieliśmy okazję nauczyć się wielu nowych rzeczy. Najciekawszym elementem według nas było korzystanie ze zdalnego repozytorium za pośrednictwem githuba. Dzięki temu praca nad kodem jest sprawniejsza, łatwiej zorganizować działania i podzielić obowiązki a przejrzystość podczas wprowadzania zmian pozwala uniknąć nieporozumień i haosu. Oprócz tego ćwiczyliśmy następujące umiejętonści: \\
	$\bullet$ pisanie kodu obiektowego w Pythonie \\
	$\bullet$ implementowanie algorytmów pochodzących ze źródeł  zewnętrznych (tj. takich, których nie wymyśliliśmy sami) \\
	$\bullet$ tworzenie dokumentów w latex \\
	$\bullet$tworzenie narzędzi w interfejsie tekstowym (cli) potrafiących przyjmować argumenty przy wywołaniu \\
	$\bullet$ pisanie użytecznej dokumentacji(chyba)
\end{flushleft}

\vspace{4cm}
\begin{center}
	\textit{Link do repozytorium:} \\
	\textbf{\url https://github.com/misiek0n/Projekt-1}
\end{center}